\documentclass[11pt,letterpaper]{article}
\usepackage[utf8]{inputenc}
\usepackage[T1]{fontenc}
\usepackage[spanish]{babel}
\usepackage{booktabs}
\usepackage{siunitx}[=v2] % La versión tres no admite Angstroms!!!
\DeclareSIUnit\Molar{\textsc{m}}
\usepackage[version=4]{mhchem} % Ecuaciones químicas.
\usepackage[dvipsnames]{xcolor}
\usepackage{hyperref}
\usepackage{cleveref}


\newcommand\myshade{85}
\colorlet{mylinkcolor}{violet}
\colorlet{mycitecolor}{YellowOrange}
\colorlet{myurlcolor}{Aquamarine}

\hypersetup{
	linkcolor  = mylinkcolor!\myshade!black,
	citecolor  = mycitecolor!\myshade!black,
	urlcolor   = myurlcolor!\myshade!black,
	colorlinks = true,
}
\author{M. C. Francisco Murphy Pérez}
\title{Avances del proyecto: Efecto del pH en la condición de
cristalización sobre el daño por radiación
en cristales de proteína}

\begin{document}
	\maketitle
	%Por favor mándame un resumen de 3-5 cuartillas en donde indiques cuál es la importancia de tu trabajo doctoral y como lo vas a desarrollar.
	\section{Introducción}	
	La cristalografía de rayos X es la técnica experimental más usada para obtener la estructura tridimensional de macromoléculas, en particular de proteínas. Los rayos X usados tienen la capacidad de ionizar las moléculas que constituyen el cristal. La ionización provoca una cascada de radicales libres que en consecuencia provocan lo que se conoce como daño por radiación. Este último se puede clasificar como daño específico y daño global. El primero produce cambios químicos en la proteína y se da en el siguiente orden: la reducción de átomos metálicos, la ruptura de puentes disulfuro, la decarboxilación de aminoácidos ácidos y la pérdida del grupo tiometilo de la metionina. El daño global se da por una pérdida en el orden cristalino \cite{Weik2000,Ravelli2000}. Sin cristal, no puede haber \emph{cristalografía de rayos X}; por lo que, en general, el daño por radiación es una de las limitantes de esta técnica experimental.
	
	\section{Antecedentes}
	Los efectos del daño por radiación específico se dan porque la muestra está en un estado vítreo a \SI{100}{\kelvin} y las especies químicas que son móviles a dicha temperatura son los electrones, los huecos positivos (la ausencia de electrón) y los átomos de hidrógeno \cite{Owen2012a}. 
	
	Por otra parte, el electrón solvatado se encuentra en un equilibrio ácido base:
	\begin{equation*}
		\ce{e^{-}_{solv.} + H^{+} <=> H^{.}}
	\end{equation*}
	El radical \ce{H^.} se puede recombinar:
	\begin{equation*}
		\ce{2H^{.}      ->        H2}
	\end{equation*} 
	Se ha comprobado que el \ce{H2} es uno de los principales productos de la radiación \cite{Meents2010}.
	
	\section{Hipótesis}
	Partimos de la hipótesis de que la alta concentración de protones, es decir, bajos niveles de pH en la condición de cristalización, permitirán mitigar el daño por radiación específico, en particular el provocado por los \ce{e^{-}_{solv.}}.
	
	\section{Objetivo}
	El objetivo de este proyecto es determinar el efecto del pH en la condición de cristalización de ciertas proteínas sobre el daño por radiación. En particular se tiene que:
	\begin{enumerate}
		\item Realizar un análisis \emph{in silico} para determinar qué proteínas son capaces de cristalizar en un intervalo de pH amplio.
		\item Cristalizar las proteínas seleccionadas a diferentes niveles de pH usando cualquiera de los sistemas de amortiguamiento que muestrean un intervalo de pH amplio
		\item Determinar los parámetros de la colecta de datos que produzcan niveles similares, o idénticos, de dosis de radiación absorbida en las diferentes proteínas cristalizadas.
		\item Realizar colectas de datos continuas y seriales en un sincrotrón.
		\item Procesar los patrones de difracción para obtener un modelo inicial de las proteínas.
		\item Mapear la diferencia de densidad electrónica entre colectas de datos al modelo inicial de cada proteína y realizar un análisis comparativo, en particular sobre los residuos de aminoácido que son más susceptibles al daño por radiación, para determinar las diferencias en daño por radiación a diferentes niveles de pH.
	\end{enumerate}	
	
	\section{Resultados}
	\begin{enumerate}
		\item Del análisis \emph{in silico} para determinar qué proteínas son capaces de cristalizar en un intervalo de pH amplio, se obtuvieron cinco proteínas con los siguientes identificadores de \verb|uniprot| \cite{Bateman2021}: \verb|P42212|, \verb|P00698|, \verb|P00760|, \verb|P00918|, \verb|P02766|. 
		\item  Con respecto a la cristalización de las proteínas seleccionadas$\ldots$ 
		\begin{itemize}
		\item Se han puesto condiciones de cristalización con la lisozima (\verb|P00698|) y la tripsina (\verb|P00760|). Ambas conseguidas de manera comercial y sin procedimientos posteriores de purificación. 
		
		\item De los sistemas de amortiguadores de capacidad extendida desarrollados por Newman \cite{Newman2004}, se utilizó la primera condición de cristalización (denominada de ahora en adelante, C1): ácido succínico, glicina y sodio dihidrógeno fosfato monohidratado a una concentración de \SI{250}{\milli\Molar}. 
		
		\item Se realizaron experimentos de cristalización con la técnica modificada de \emph{microbatch}, propuesta por D'Arcy \emph{et al.} \cite{DArcy1996}. En estos experimentos, se varió la concentración de la proteína (15 o \SI{30}{\milli\gram\per\milli\liter}); la proporción entre aceites de parafina y silicona (1:0, 1:0.5, 1:1, 1:2 y 1:4); el pH (10, 9.5, 9.0, $\ldots$, 4.5) y finalmente, también se cambió el porcentaje de cloruro de sodio (0, 3 y \SI{6}{\percent} p/v) en la condición de cristalización. Esto debido a que, en ciertas condiciones, se ha observado una dependencia del tamaño de los cristales de lisozima obtenidos con respecto a la concentración de cloruro de sodio \cite{Svanidze2005}. Se usaron placas de cristalización de Terasaki Greiner de \num{72} pozos. Esto permite tres corridas de pH, con tres porcentajes de cloruro de sodio por duplicado. Cabe destacar que en cada placa la proporción de aceites es diferente. En total, se tienen \num{720} experimentos (\num{1440}, tomando en cuenta los duplicados). Esto último, tiene que ver con la reproducibilidad de los resultados de los experimentos de cristalización, pues no siempre son idénticos.
		
		\item Para clasificar los resultados de cristalización, se usó un esquema de clasificación modificado a partir del esquema de Bruno \emph{et al.} \cite{Bruno2018}. La clasifiación es la siguiente: 1 (cristales), 0.5 (cristales pequeños), 0 (gota clara), -1 (precipitado) y -2 (otros). Se tomaron tres fotos con una cámara USB acoplada a un microscopio óptico en los días quinto, noveno y décimosexto. Esto significa que se visualizaron y clasificaron \num{4320} fotos. Los resultados se definieron como positivos si se presentan cristales, en al menos uno de los dos experimentos, en los siguientes valores de pH 10, 9, 7.5, 5.0 y 4.5; de lo contrario, se consideran como resultados negativos. Para la tripsina, por lo menos al momento del día de la tercera foto, todos los resultados fueron negativos. Los resultados positivos para la lisozima se resumen a continuación.
		
		
		\begin{table}[h]
			\centering
			\begin{tabular}{@{}llll@{}}
				\toprule
				Proteína & Concentración & \% p/v NaCl & AP:AS \\ \midrule
				Lisozima & 15 & 3 & 1:1 \\
				Lisozima & 15 & 3 & 1:4 \\
				Lisozima & 15 & 6 & 1:4 \\
				Lisozima & 30 & 3 & 1:1 \\
				Lisozima & 30 & 6 & 1:1 \\
				Lisozima & 30 & 3 & 1:2 \\
				Lisozima & 30 & 6 & 1:2 \\
				Lisozima & 30 & 3 & 1:4 \\
				Lisozima & 30 & 6 & 1:4 \\ \bottomrule
			\end{tabular}
			\caption{Condiciones de cristalización exitosas para la lisozima. AP y AS significan aceite de parafina y silicona, respectivamente.}
			\label{tab:my-table}
		\end{table}
	\end{itemize}

	\item Con respecto a determinar los parámetros de la colecta de datos que produzcan niveles similares, o idénticos, de dosis de radiación absorbida, se escribió un programa en el lenguaje de programación \verb|bash| que permite obtener la dosis de radiación absorbida,  calculada por \verb|raddose| \cite{Bury2018}. Por simplicidad, se determinó un cristal hipotético de lisozima cúbico, con un volumen de \SI{1000000}{\cubic\angstrom}. Este presenta la misma composición atómica que la entrada \verb|1iee| del \verb|PDB|. También por simplicidad, el haz de rayos X se definió como colimado, con un corte de \num{100} x  \SI{100}{\micro\meter}, abarcando el área completa del cristal y un tiempo total de exposición de \SI{100}{\second}, con la obtención de un patrón de difracción por segundo. Dicho programa muestrea un flujo de fotones de 1.0 a \SI{100e11}{fotones\per\second} con un paso d \SI{0.1}{fotones\per\second} y una energía de 10.0 a \SI{17.2}{\kilo\electronvolt} con un paso de \SI{0.2}{\kilo\electronvolt}. Se obtuvo acceso al clúster \verb|groc| del grupo de genómica computacional del Instituto de Biotecnología de la UNAM\footnote{\url{https://biocomputo.ibt.unam.mx/}.}para correr dicho programa. Cabe aclarar que si bien los parámetros experimentales dados son idóneos, el fin de este programa es tener un estimado de la dosis por radiación absorbida en un sincrotrón de tercera generacion. Por otra parte, este primer programa sienta las bases para cualquier programa posterior en el que se necesite adaptar los parámetros experimentales a la realidad.
	 
	\item Con respecto a las colectas de datos, como primer acercamiento, se llevaron cinco\footnote{Crecidos a pH 10 (dos), 7.5 (uno) y 4.5 (dos).} cristales de lisozima al Laboratorio Nacional de Estructura de Macromoléculas (LANEM) del Instituto de Química (IQ) de la UNAM el 20 de mayo del presente año. En el LANEM-IQ, se tiene acceso a un generador de rayos X \verb|MicroMax-007 HF| de \verb|Rigaku|\footnote{\url{https://www.rigaku.com/products/sources/mm007}}. Hasta el momento, el único dato del que se tiene conocimiento es que la difracción de al menos uno de los cristales era mejor que \SI{1.6}{\angstrom}. 
	\end{enumerate}
	 
 	\section{Perspectivas}
 	
 	\subsection{Tripsina}
 	Los resultados negativos con la tripsina podrían ser por: (\emph{i}) la concentración del agente precipitante, de la proteína o de ambas es muy baja. O (\emph{ii}) la nucleación y en consecuencia la formación de cristales es menor, comparada con la de la lisozima. Es necesario, tomar una última ronda de fotos de las placas de cristalización de tripsina para descartar o confirmar la segunda razón. De igual manera, para descartar o confirmar la primera razón, es necesario poner nuevas condiciones de cristalización con mayor concentración de proteína, de agente precipitante o de ambos y además utilizar los otros sistemas de amortiguadores de capacidad extendida.
 	
 	\subsection{Lisozima}
 	Si bien no se tienen todos los detalles de los experimentos de difracción en el ánodo rotatorio del LANEM-IQ, lo que se sabe con certeza es que al menos unos de los cristales difractados era de buena calidad. Aunado al número de condiciones exitosas para la formación de cristales, esto implica que no es necesario muestrear más condiciones para la cristalización de esta proteína; sino replicar las condiciones que se encontraron.
 	
 	\subsection{Dosis de radiación}
 	Los resultados del programa para determinar la dosis de radiación en un sincrotrón de tercera generación, muestreando múltiples flujos de fotones y energías, no se han obtenido. Debido a esto, se realizó una estimación de la dosis de radiación absorbida para un cristal de lisozima hipotético\footnote{Con las mismas características, que el mencionado en el texto principal.}, bajo las condiciones del ánodo rotatorio del LANEM-IQ. Suponiendo una exposición de un segundo por cada patrón de difracción y rotando el cristal cien grados para obtener cien patrones de difracción, se obtiene una dosis promedio de radiación absorbida de \SI{0.000132}{\mega\gray}. Si se cambia el tiempo total de exposición de \SI{100}{\second} a \SI{1000}{\second}, se obtiene como resultado un aumento en la dosis por un factor de diez. Esto significa que se necesitarían cerca de mil colectas idénticas para alcanzar una dosis promedio de \SI{1.3}{\mega\gray} o \SI{13}{\mega\gray}, dependiendo del tiempo de exposición selecccionado. Suponiendo la ausencia de tiempos muertos entre colectas, la colecta de un cristal de lisozima tardaría aproximadamente 11.5 días en el LANEM-IQ. Si queremos difractar tres cristales crecidos a tres valores de pH (10, 7.5 y 4.5), es decir, nueve experimentos, esto resulta en un tiempo de uso cerca de 100 días. Demasiado tiempo, comparado con una colecta en un sincrotrón de tercera generación.
	 
	  %El procesamiento de los patrones de difracción se realizará de manera automática con un script creado previamente\footnote{https://github.com/murpholinox/img2mtz}. La obtención del modelo estructural viene de una simple sustitución molecular, este modelo se afinará con \verb|phenix.refine| y \verb|coot| \cite{Liebschner2019,Emsley2010}. La diferencia positiva y negativa de densidad electrónica entre las colectas de datos, se obtendrá con el programa \verb|ridl| \cite{Bury2018a}. La comparación del daño estructural a diferentes niveles de pH, podrá indicarnos si existe algún efecto del pH sobre el daño por radiación, confirmando o descartando nuestra hipótesis.
	
	\bibliographystyle{unsrt}
	\bibliography{bib}
\end{document}