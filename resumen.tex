\documentclass[11pt,letterpaper]{article}
\usepackage[utf8]{inputenc}
\usepackage[T1]{fontenc}
\usepackage[spanish]{babel}
\usepackage{booktabs}
\usepackage{siunitx}[=v2] % La versión tres no admite Angstroms!!!
\DeclareSIUnit\Molar{\textsc{m}}
\usepackage[version=4]{mhchem} % Ecuaciones químicas.
\usepackage[dvipsnames]{xcolor}
\usepackage{hyperref}
\usepackage{cleveref}
\usepackage{longtable}


\newcommand\myshade{85}
\colorlet{mylinkcolor}{violet}
\colorlet{mycitecolor}{YellowOrange}
\colorlet{myurlcolor}{Aquamarine}

\hypersetup{
	linkcolor  = mylinkcolor!\myshade!black,
	citecolor  = mycitecolor!\myshade!black,
	urlcolor   = myurlcolor!\myshade!black,
	colorlinks = true,
}
\author{M. C. Francisco Murphy Pérez}
\title{Avances del proyecto: Efecto del pH en la condición de
cristalización sobre el daño por radiación
en cristales de proteína}

\begin{document}
\maketitle
%Por favor mándame un resumen de 3-5 cuartillas en donde indiques cuál es la importancia de tu trabajo doctoral y como lo vas a desarrollar.
\section{Introducción}	
La cristalografía de rayos-X es la técnica experimental más usada para obtener la estructura tridimensional de proteínas. Dicha técnica está basada en la difracción de rayos-X, por los electrones de los átomos que constituyen el cristal de proteína. Los rayos-X usados tienen la energía suficiente para ionizar los átomos que constituyen el cristal de proteína. La ionización provoca una cascada de radicales libres que a su vez provocan lo que se conoce como daño por radiación. El daño por raidiación se clasifica como daño específico y daño global. El primero produce cambios químicos en la proteína y se da en el siguiente orden: la reducción de átomos metálicos, la ruptura de puentes disulfuro, la descarboxilación de aminoácidos ácidos y la pérdida del grupo tiometilo de la metionina. Por otra parte, el daño global se da por una pérdida en el orden cristalino, lo que genera varias consecuencias en los datos experimentales obtenidos \cite{Weik2000,Ravelli2000}. En general, el daño por radiación es una de las limitantes de la cristalografía de rayos-X.
	
\section{Antecedentes}
Los efectos del daño por radiación específico se dan porque la muestra está en un estado vítreo a \SI{100}{\kelvin} y las especies químicas que son móviles a dicha temperatura son los electrones, los huecos positivos (la ausencia de electrón) y los átomos de hidrógeno \cite{Owen2012a}. 
	
Por otra parte, el electrón solvatado se encuentra en un equilibrio ácido base:
\begin{equation*}
\ce{e^{-}_{solv.} + H^{+} <=> H^{.}}
\end{equation*}

El radical \ce{H^.} se puede recombinar:
\begin{equation*}
\ce{2H^{.}      ->        H2}
\end{equation*} 
Se ha comprobado que el \ce{H2} es uno de los principales productos de la radiación \cite{Meents2010}.
	
\section{Hipótesis}
La alta concentración de protones, es decir, bajos niveles de pH en la condición de cristalización, permitirán mitigar el daño por radiación específico, en particular el provocado por los \ce{e^{-}_{solv.}}.
	
\section{Objetivo}
El objetivo de este proyecto es determinar el efecto del pH en la condición de cristalización de ciertas proteínas sobre el daño por radiación. En particular se tiene que:

\begin{enumerate}
\item Realizar un análisis \emph{in silico} para determinar qué proteínas son capaces de cristalizar en un intervalo de pH amplio.
\item Cristalizar las proteínas seleccionadas a diferentes niveles de pH usando cualquiera de los sistemas de amortiguamiento que muestrean un intervalo de pH amplio
\item Determinar los parámetros de la colecta de datos que produzcan niveles similares, o idénticos, de dosis de radiación absorbida en las diferentes proteínas cristalizadas.
\item Realizar colectas de datos idénticas y continuas para cada cristal en un sincrotrón.
\item Procesar los patrones de difracción para obtener un modelo inicial de las proteínas en estado vítreo.
\item Mapear la diferencia de densidad electrónica entre colectas de datos al modelo inicial de cada proteína y realizar un análisis comparativo, en particular sobre los residuos de aminoácido que son más susceptibles al daño por radiación, para determinar las diferencias en daño por radiación a diferentes niveles de pH.
\end{enumerate}	
	
\section{Resultados}
\subsection{Proteínas capaces de cristalizar a diferentes valores de pH}
Se rehizo el análisis \emph{in silico} para determinar qué proteínas son capaces de cristalizar en un intervalo de pH amplio (código \href{https://github.com/murpholinox/lap_book}{en este enlace}). 
\begin{table}[h]
\centering
\resizebox{\textwidth}{!}{%
\begin{tabular}{@{}lllll@{}}
\toprule
Identificador & Grupo espacial & Intervalo de pH & Organismo         & Proteína                   \\ \midrule
P42212        & P 21 21 21     & 7               & \emph{Aequorea victoria} & Green fluorescent protein  \\
P01116        & P 21 21 21     & 6.5             & \emph{Homo sapiens}      & GTPase Kras                \\
P00760        & P 31 2 1       & 6.15            & \emph{Bos taurus}        & Serine protease 1          \\
P00918        & P 1 21 1       & 5.9             & \emph{Homo sapiens}      & Carbonic anhydrase 2       \\
P00698        & P 43 21 2      & 5.5             & \emph{Gallus gallus}     & Lysozyme C                 \\
P00772        & P 21 21 21     & 5.4             & \emph{Sus scrofa}        & Chymotrypsin-like elastase \\
P02766        & P 21 21 2      & 5               & \emph{Homo sapiens}      & Transthyretin              \\
P00760        & P 21 21 21     & 4.5             & \emph{Bos taurus}        & Serine protease 1          \\ \bottomrule
\end{tabular}%
}
\caption{Proteínas capaces de cristalizar en un intervalo de pH amplio.}
\label{tab:lap}
\end{table}
\subsection{Cristalización de proteínas}
Hasta el momento se han puesto pruebas de cristalización con lisozima y tripsina; ambas conseguidas de manera comercial y sin procedimientos posteriores de purificación (número de catálogo \verb|L6876| y \verb|T8003|, respectivamente). También se pusieron pruebas con las proteínas ornitina descarboxilasa 1 y 2 (ODC1/2) y con la variante de la proteína verde fluorescente denominada PF17-8. Las primeras dos en colaboración con el M. C. Aldo Emmanuel Pérez Rivera y la tercera en colaboración con el Dr. Victor Rivelino Juárez González. 

De los siguientes sistemas de amortiguadores de capacidad extendida, desarrollados por Newman \cite{Newman2004}, se utilizaron los siguientes: 
\begin{itemize}
\item Sistema uno (S01); ácido succínico, glicina y sodio dihidrógeno fosfato monohidratado.
\item Sistema dos (S02); ácido cítrico, HEPES y CHES.
\item Sistema cuatro (S04); propionato de sodio, cacodilato de sodio trihidratado, BIS-TRIS propano.
\item Sistema seis (S06); ácido L-málico, MES y TRIS.
\end{itemize}
Todos a una concentración de \SI{250}{\milli\Molar} (\SI{125}{\milli\Molar}, en la gota de cristalización) y con \SI{3}{\percent} peso/volumen de cloruro de sodio como agente precipitante. 
		
Las pruebas de cristalización se realizaron por medio de la técnica modificada de \emph{microbatch}, propuesta por D'Arcy \emph{et al.} \cite{DArcy1996}. En estos experimentos, se varió la concentración de la proteína (15 o \SI{30}{\milli\gram\per\milli\liter}) y/o la proporción entre aceites de parafina y silicona (1:0.5, 1:1, 1:2) y el pH (10, 9.5, 9.0, $\ldots$, 4.5). Se usaron placas de cristalización de Terasaki Greiner de \num{72} pozos que permiten poner tres corridas de pH (triplicado), por sistema de amortiguamiento (dos por placa). Cabe destacar que en cada placa la proporción de aceites es diferente. En total, se pusieron \num{28} placas como se detalla en los siguientes cuadros. Finalmente, estas placas se mantuvieron a \SI{18}{\degreeCelsius}.
				
\begin{table}[h]
\centering
\resizebox{0.8\columnwidth}{!}{%
\begin{tabular}{@{}lllll@{}}
\toprule
Placa & Proteína & Conc. (\si{\milli\gram\per\milli\liter}) & PO:SO & Sistemas \\
\midrule
1      & tripsina & 50            & 1:0.5 & S01, S06 \\
2      & tripsina & 50            & 1:1   & S01, S06 \\
3      & tripsina & 50            & 1:2   & S01, S06 \\
4      & lisozima & 30            & 1:0.5 & S01, S06 \\
5      & lisozima & 30            & 1:1   & S01, S06 \\
6      & lisozima & 30            & 1:2   & S01, S06 \\
7      & tripsina & 50            & 1:0.5 & S02, S04 \\
8      & tripsina & 50            & 1:1   & S02, S04 \\
9      & tripsina & 50            & 1:2   & S02, S04 \\
10     & lisozima & 30            & 1:0.5 & S02, S04 \\
11     & lisozima & 30            & 1:1   & S02, S04 \\
12     & lisozima & 30            & 1:2   & S02, S04 \\
\bottomrule
\end{tabular}%
}
\caption{Pruebas de cristalización de lisozima y tripsina. Gotas de \SI{4}{\micro\liter}.}
\label{tab:crislisoytrip}
\end{table}		
			
\begin{table}[h]
\centering
\resizebox{0.8\columnwidth}{!}{%
\begin{tabular}{@{}lllll@{}}
\toprule
Placa & Proteína & Conc. (\si{\milli\gram\per\milli\liter}) & PO:SO & Sistemas \\
\midrule
13 & ODC2 & 38  & 1:0.5 & S01, S06 \\
14 & ODC2 & 38  & 1:0.5 & S02, S04 \\
15 & ODC1 & 2.1 & 1:0.5 & S01, S06 \\
16 & ODC1 & 2.1 & 1:0.5 & S02, S04 \\
\bottomrule
\end{tabular}%
}
\caption{Pruebas de cristalización de ODC1 y ODC2. Gotas de \SI{4}{\micro\liter}. }
\label{tab:crisodcs}
\end{table}	
		
\begin{table}[h]
\centering
\resizebox{0.8\columnwidth}{!}{%
\begin{tabular}{@{}lllll@{}}
\toprule
Placa & Proteína & Conc. (\si{\milli\gram\per\milli\liter}) & PO:SO & Sistemas \\
\midrule
19 & PF17-8 & 4.8  & 1:1 & S01, S06 \\
20 & PF17-8 & 4.8  & 1:2 & S02, S04 \\
21 & PF17-8 & 4.8  & 1:1 & S01, S06 \\
22 & PF17-8 & 4.8  & 1:2 & S02, S04 \\
23 & PF17-8 & 12.9 & 1:1 & S01, S06 \\
24 & PF17-8 & 12.9 & 1:2 & S02, S04 \\
25 & PF17-8 & 12.9 & 1:1 & S01, S06 \\
26 & PF17-8 & 12.9 & 1:2 & S02, S04 \\
27 & PF17-8 & 30   & 1:1 & S02, S04 \\
28 & PF17-8 & 30   & 1:1 & S01, S06 \\
29 & PF17-8 & 30   & 1:2 & S02, S04 \\
30 & PF17-8 & 30   & 1:2 & S01, S06 \\
\bottomrule
\end{tabular}%
}
\caption{Pruebas de cristalización de PF17-8. Gotas de \SI{4}{\micro\liter}.}
\label{tab:crispf178}
\end{table}

Además se pusieron 2 placas con más pruebas de cristalización, con la técnica conocida como gota colgante, en cajas de cristalización de 24 pozos. Las pruebas en este aspecto, tenían dos sentidos; primero, tratar de reproducir las condiciones en las que se sabe que la lisozima genera cristales en los valores de pH extremos y segundo, ver la diferencia en temperatura. El sistema de amortiguamiento usado fue el sistema cuatro (S04), la disposición de las condiciones de cristalización en ambas cajas fue la siguiente: cada fila contiene los siguientes valores de pH, \num{4}, \num{4.5}, \num{5}, \num{8.5}, \num{9} y \num{9.5}; por otra parte, cada fila tiene una mezcla de aceites de parafina y silicona diferentes (en un intento por reproducir las condiciones de los experimentos de cristalización de \emph{microbatch}). Una de las placas se mantuvo a \SI{18}{\degreeCelsius} y la otra a \SI{4}{\degreeCelsius}. 

\subsection{Resultados de cristalización}
Se tomaron fotos con una cámara USB acoplada a un microscopio óptico cada diez días, hasta cuarenta días. Hasta el momento se han tomado 16 sesiones de fotos y restan cinco sesiones. El esquema de clasificación usado anteriormente; se cambió, por simplicidad, con el siguiente esquema: cristales bien definidos, posibles cristales, sin cristales.
\begin{enumerate}
	\item Las pruebas de cristalización de tripsina han dado cristales bien definidos solo en el sistema uno y a valores bajos de pH. Hasta el momento se tienen diez cristales a un pH de: 4.5 (1), 5 (6) y 5.5 (3). La mayor parte de los cristales presentan una morfología planar en forma de rombo.
	\item Las pruebas de cristalización de lisozima han dado cristales bien definidos en la mayoría de los sistemas (uno, cuatro y seis); sin embargo, el sistema donde se generan más cristales, en promedio, es el sistema cuatro. En particular, hasta diez de los doce pozos pueden presentar resultados favorables. Los cristales aparecen en múltiples formas; cúbicos, hexagonales o romboides.  
	\item Las pruebas de cristalización de ODC1 fueron negativas en su totalidad. Con respecto a ODC2, parece haber cristales en el sistema uno a valores medios de pH. Hasta el momento se tienen cinco posibles cristales a un pH de: \num{6.5} (\num{1}), \num{7.5} (\num{3}) y \num{9.5} (\num{1}). El cristal a pH 6.5 es el único en el sistema seis.
	\item Las pruebas de cristalización de PF17-8 han producido seis cristales. Uno a pH \num{5} en el sistema uno; uno a pH \num{5} y otro a pH \num{7} en el sistema dos; y finalmente, uno a pH \num{4.5} y dos a pH \num{7.5}. Estos cristales no presentan una morfología tan regular, a comparación de los de lisozima, pero ya se ha comprabado que son cristales de proteína.
\end{enumerate}

\subsection{Afinamiento de las estructuras obtenidas}
Con respecto a las colectas de datos realizadas el 21 de julio, como primer acercamiento, se llevaron cinco\footnote{Crecidos a pH 10 (dos), 7.5 (uno) y 4.5 (dos).} cristales de lisozima al Laboratorio Nacional de Estructura de Macromoléculas (LANEM) del Instituto de Química (IQ) de la UNAM el 20 de mayo del presente año. En el LANEM-IQ, se tiene acceso a un generador de rayos X \verb|MicroMax-007 HF| de \verb|Rigaku|\footnote{\url{https://www.rigaku.com/products/sources/mm007}}. Hasta el momento, el único dato del que se tiene conocimiento es que la difracción de al menos uno de los cristales era mejor que \SI{1.6}{\angstrom}. 
	 
\section{Perspectivas}
 	
\subsection{Tripsina}

\subsection{Lisozima}
 	
\subsection{Dosis de radiación}
	 
%El procesamiento de los patrones de difracción se realizará de manera automática con un script creado previamente\footnote{https://github.com/murpholinox/img2mtz}. La obtención del modelo estructural viene de una simple sustitución molecular, este modelo se afinará con \verb|phenix.refine| y \verb|coot| \cite{Liebschner2019,Emsley2010}. La diferencia positiva y negativa de densidad electrónica entre las colectas de datos, se obtendrá con el programa \verb|ridl| \cite{Bury2018a}. La comparación del daño estructural a diferentes niveles de pH, podrá indicarnos si existe algún efecto del pH sobre el daño por radiación, confirmando o descartando nuestra hipótesis.
	
\bibliographystyle{unsrt}
\bibliography{library}
\end{document}